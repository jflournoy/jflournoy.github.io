% Options for packages loaded elsewhere
\PassOptionsToPackage{unicode}{hyperref}
\PassOptionsToPackage{hyphens}{url}
\PassOptionsToPackage{dvipsnames,svgnames,x11names}{xcolor}
%
\documentclass[
  10pt,
  letterpaper,
  DIV=11,
  numbers=noendperiod]{scrartcl}

\usepackage{amsmath,amssymb}
\usepackage{iftex}
\ifPDFTeX
  \usepackage[T1]{fontenc}
  \usepackage[utf8]{inputenc}
  \usepackage{textcomp} % provide euro and other symbols
\else % if luatex or xetex
  \usepackage{unicode-math}
  \defaultfontfeatures{Scale=MatchLowercase}
  \defaultfontfeatures[\rmfamily]{Ligatures=TeX,Scale=1}
\fi
\usepackage{lmodern}
\ifPDFTeX\else  
    % xetex/luatex font selection
\fi
% Use upquote if available, for straight quotes in verbatim environments
\IfFileExists{upquote.sty}{\usepackage{upquote}}{}
\IfFileExists{microtype.sty}{% use microtype if available
  \usepackage[]{microtype}
  \UseMicrotypeSet[protrusion]{basicmath} % disable protrusion for tt fonts
}{}
\makeatletter
\@ifundefined{KOMAClassName}{% if non-KOMA class
  \IfFileExists{parskip.sty}{%
    \usepackage{parskip}
  }{% else
    \setlength{\parindent}{0pt}
    \setlength{\parskip}{6pt plus 2pt minus 1pt}}
}{% if KOMA class
  \KOMAoptions{parskip=half}}
\makeatother
\usepackage{xcolor}
\setlength{\emergencystretch}{3em} % prevent overfull lines
\setcounter{secnumdepth}{-\maxdimen} % remove section numbering
% Make \paragraph and \subparagraph free-standing
\makeatletter
\ifx\paragraph\undefined\else
  \let\oldparagraph\paragraph
  \renewcommand{\paragraph}{
    \@ifstar
      \xxxParagraphStar
      \xxxParagraphNoStar
  }
  \newcommand{\xxxParagraphStar}[1]{\oldparagraph*{#1}\mbox{}}
  \newcommand{\xxxParagraphNoStar}[1]{\oldparagraph{#1}\mbox{}}
\fi
\ifx\subparagraph\undefined\else
  \let\oldsubparagraph\subparagraph
  \renewcommand{\subparagraph}{
    \@ifstar
      \xxxSubParagraphStar
      \xxxSubParagraphNoStar
  }
  \newcommand{\xxxSubParagraphStar}[1]{\oldsubparagraph*{#1}\mbox{}}
  \newcommand{\xxxSubParagraphNoStar}[1]{\oldsubparagraph{#1}\mbox{}}
\fi
\makeatother


\providecommand{\tightlist}{%
  \setlength{\itemsep}{0pt}\setlength{\parskip}{0pt}}\usepackage{longtable,booktabs,array}
\usepackage{calc} % for calculating minipage widths
% Correct order of tables after \paragraph or \subparagraph
\usepackage{etoolbox}
\makeatletter
\patchcmd\longtable{\par}{\if@noskipsec\mbox{}\fi\par}{}{}
\makeatother
% Allow footnotes in longtable head/foot
\IfFileExists{footnotehyper.sty}{\usepackage{footnotehyper}}{\usepackage{footnote}}
\makesavenoteenv{longtable}
\usepackage{graphicx}
\makeatletter
\def\maxwidth{\ifdim\Gin@nat@width>\linewidth\linewidth\else\Gin@nat@width\fi}
\def\maxheight{\ifdim\Gin@nat@height>\textheight\textheight\else\Gin@nat@height\fi}
\makeatother
% Scale images if necessary, so that they will not overflow the page
% margins by default, and it is still possible to overwrite the defaults
% using explicit options in \includegraphics[width, height, ...]{}
\setkeys{Gin}{width=\maxwidth,height=\maxheight,keepaspectratio}
% Set default figure placement to htbp
\makeatletter
\def\fps@figure{htbp}
\makeatother

\usepackage{fontspec}
\usepackage{geometry}
\usepackage{fontawesome5}
\usepackage{enumitem}
\usepackage{etoolbox}
\usepackage{changepage}
\usepackage{hyperref}
\defaultfontfeatures{Ligatures=TeX, Contextuals=Alternate}

% Adjust page margins to be more compact
\geometry{margin=0.75in}

% MAIN (SERIF) FONT: Cormorant
\setmainfont[
  Path = fonts/,
  UprightFont    = {Cormorant-Regular.ttf},
  ItalicFont     = {Cormorant-Italic.ttf},
  BoldFont       = {Cormorant-Bold.ttf},
  BoldItalicFont = {Cormorant-BoldItalic.ttf}
]{Cormorant}

% SANS FONT: Montserrat
\setsansfont[
  Path = fonts/,
  UprightFont    = {Montserrat-Regular.ttf},
  ItalicFont     = {Montserrat-Italic.ttf},
  BoldFont       = {Montserrat-Bold.ttf},
  BoldItalicFont = {Montserrat-BoldItalic.ttf}
]{Montserrat}

% MONO FONT: FiraCode
\setmonofont[
  Path = fonts/,
  UprightFont = {FiraCode-Regular.ttf},
  BoldFont    = {FiraCode-Bold.ttf},
  Ligatures   = TeX,
  Contextuals = Alternate
]{Fira Code}

% Headings in sans
\usepackage{sectsty}
\allsectionsfont{\sffamily}

% Custom entry command for job titles with dates aligned right
\newcommand{\entry}[2]{%
  \noindent \textbf{#1} \hfill {\small #2}
}

% Customize all itemize environments globally
\setlist[itemize]{
  leftmargin=1em,
  itemsep=0.2em,
  parsep=0em,
  topsep=0.3em,
  rightmargin=10em
}

% Additional customization for itemize in work experience sections
\BeforeBeginEnvironment{itemize}{\vspace{0.1em}\setlength{\parindent}{0pt}}
\AfterEndEnvironment{itemize}{\vspace{0.4em}}

% Custom fancy bullet styling for fit summary section
\usepackage{xcolor}
\definecolor{accentcolor}{RGB}{70, 130, 180}

% Define a custom bullet bullet style for the fit summary section
\newcommand{\fancybullet}{\textcolor{accentcolor}{\faCheckCircle}\ }

% Create a local group for fancy bullets to avoid affecting other sections
\newcommand{\fancybulletsstart}{%
  \bgroup%
  \setlist[itemize]{%
    label=\textcolor{accentcolor}{\faCheckCircle},
    leftmargin=1.3em,
    itemsep=0.4em,
    parsep=0em,
    topsep=0.3em,
    labelsep=0.5em
  }%
}
\newcommand{\fancybulletsend}{\egroup}

% Setup hyperlinks in blue
\hypersetup{colorlinks=true, urlcolor=blue}
\KOMAoption{captions}{tableheading}
\makeatletter
\@ifpackageloaded{caption}{}{\usepackage{caption}}
\AtBeginDocument{%
\ifdefined\contentsname
  \renewcommand*\contentsname{Table of contents}
\else
  \newcommand\contentsname{Table of contents}
\fi
\ifdefined\listfigurename
  \renewcommand*\listfigurename{List of Figures}
\else
  \newcommand\listfigurename{List of Figures}
\fi
\ifdefined\listtablename
  \renewcommand*\listtablename{List of Tables}
\else
  \newcommand\listtablename{List of Tables}
\fi
\ifdefined\figurename
  \renewcommand*\figurename{Figure}
\else
  \newcommand\figurename{Figure}
\fi
\ifdefined\tablename
  \renewcommand*\tablename{Table}
\else
  \newcommand\tablename{Table}
\fi
}
\@ifpackageloaded{float}{}{\usepackage{float}}
\floatstyle{ruled}
\@ifundefined{c@chapter}{\newfloat{codelisting}{h}{lop}}{\newfloat{codelisting}{h}{lop}[chapter]}
\floatname{codelisting}{Listing}
\newcommand*\listoflistings{\listof{codelisting}{List of Listings}}
\makeatother
\makeatletter
\makeatother
\makeatletter
\@ifpackageloaded{caption}{}{\usepackage{caption}}
\@ifpackageloaded{subcaption}{}{\usepackage{subcaption}}
\makeatother

\ifLuaTeX
  \usepackage{selnolig}  % disable illegal ligatures
\fi
\usepackage{bookmark}

\IfFileExists{xurl.sty}{\usepackage{xurl}}{} % add URL line breaks if available
\urlstyle{same} % disable monospaced font for URLs
\hypersetup{
  pdfauthor={John C. Flournoy, PhD},
  colorlinks=true,
  linkcolor={blue},
  filecolor={Maroon},
  citecolor={Blue},
  urlcolor={Blue},
  pdfcreator={LaTeX via pandoc}}


\author{}
\date{}

\begin{document}

% Custom title with less spacing
\begin{center}
{\LARGE\sffamily\textbf{John C. Flournoy, PhD}\par}
\vspace{0.2em}
{\large\sffamily Quantitative Research Scientist \& Methodological Consultant\par}
\vspace{0.3em}

% Contact information with Font Awesome icons
\sffamily\small
\faPhone\ (415) 260-2744 \quad
\faEnvelope\ \href{mailto:jcflournoyphd@pm.me}{jcflournoyphd@pm.me} \quad
\faGlobe\ \href{https://johnflournoy.science}{johnflournoy.science} \quad
\faGraduationCap\ \href{https://scholar.google.com/citations?user=ZzQlngkAAAAJ&hl=en}{Google Scholar}
\end{center}
\vspace{0.3em}


\section{Professional Summary}\label{professional-summary}

Quantitative methodologist and research scientist with 12+ years
translating complex statistical problems into practical guidance for
multidisciplinary teams. I combine deep technical expertise with
substantive research experience in developmental science, education, and
learning---and I build the tools, workflows, and capacity that help
teams do rigorous work independently.

\section{Why This Role: Fit Summary}\label{why-this-role-fit-summary}

\fancybulletsstart

\begin{itemize}
\item
  Deep expertise in causal inference, Bayesian methods, longitudinal
  analysis, measurement theory, and study design---the quantitative
  methods central to rigorous education research and impact evaluation
\item
  Proven mentoring and capacity building: 12+ years building
  quantitative skills within multidisciplinary teams through workshops,
  one-on-one consultation, and reusable tools and templates;
  demonstrated ability to teach rigorous methods to collaborators across
  varying statistical backgrounds
\item
  Youth-centered research background: Extensive experience designing and
  conducting research with adolescents, including vulnerable populations
  (foster-care-involved youth) requiring enhanced ethical protections;
  deep knowledge of measurement validity and implementation
  considerations specific to youth
\item
  Direct education and learning research: Designed and evaluated
  pedagogical interventions at a learning technology company---testing
  whether framing assessment as learning improved engagement, and
  whether communities of practice supported persistence
\item
  Substantive breadth rooted in cognitive and developmental science:
  Core expertise in cognition, development, and psychometrics extends
  naturally into mental health, education technology, and learning
  research---enabling methodological support across foundry10's diverse
  labs and interest areas
\item
  Committed advocate and practitioner of open science: Published on
  improving scientific practices in developmental research; proponent of
  multiverse analysis, pre-registration, and transparent reporting;
  build and share reusable research tools (R packages, Docker
  environments, survey scoring pipelines) and apply test-driven
  development to analytic workflows
\end{itemize}

\fancybulletsend

\section{Core Technical Expertise}\label{core-technical-expertise}

\textbf{Advanced Statistical \& Causal Methods}

Multilevel/hierarchical linear modeling · Longitudinal data analysis
(mixed effects, latent growth curves, intensive designs) · Bayesian
inference \& modeling (Stan, brms, hierarchical models, prior
specification, posterior diagnostics) · Causal inference (DAGs,
confounding adjustment, sensitivity analysis, instrumental variables) ·
Structural equation modeling · Network modeling · Measurement theory \&
psychometrics (IRT, factor analysis, measurement invariance,
reliability)

\textbf{Study Design \& Methodology}

Experimental design \& power analysis · Observational study design
(matching, weighting, stratification) · Survey sampling \& weighted data
analysis · Mixed-methods integration · Research protocols \&
pre-registration · Human subjects research with youth and vulnerable
populations · Missing data methods (maximum likelihood, multiple
imputation)

\textbf{Computational \& Reproducibility Skills}

R (data.table, ggplot2, brms, Stan) · Python · SQL · Git/GitHub · R
Markdown \& Quarto · Open science practices (OSF, pre-registration,
transparent reporting, multiverse analysis)

\section{Professional Experience}\label{professional-experience}

\entry{Research and Data Science Consultant, Independent}{2018--Present}

\begin{itemize}
\tightlist
\item
  Provide expert statistical consultation to academic research teams and
  organizations on study design, causal inference approaches,
  measurement validation, and Bayesian modeling
\item
  Clients include Harvard neuroimaging labs, digital mental health
  platforms (Meru Health), and academic teams across psychology and
  education
\item
  Design observational longitudinal studies with attention to data
  collection strategies, missing data approaches, and hierarchical model
  specification
\item
  Validate and develop measurement instruments through item analysis,
  internal consistency evaluation, and structural validity testing (SEM,
  IRT)
\item
  Maintain active self-directed learning and experimental projects
  (i.e., play) across statistical, computational, and visualization
  domains: Bayesian modeling, ML pipelines, visualization tools, and
  generative AI experiments (see
  \href{https://johnflournoy.science/projects}{johnflournoy.science/projects})
\end{itemize}

\entry{Principal Research Scientist, Developer Success Lab, Pluralsight}{2024--2025}

\begin{itemize}
\tightlist
\item
  Led research spanning learning science and developer productivity at a
  company with two products: Skills (a learning platform) and Flow
  (productivity analytics)
\item
  Designed and analyzed microstudies evaluating pedagogical
  interventions: whether framing testing as learning (vs.~evaluation)
  improved learner engagement, and whether communities of practice
  supported persistence in learning
\item
  Led statistical analysis of longitudinal productivity data from
  11,000+ developers across 216 organizations (55,000+ observations);
  specified multilevel Bayesian models separating individual from
  organizational variation
\item
  Published peer-reviewed evaluation demonstrating that
  individual-focused interventions have limited system-level impact,
  informing strategic shift toward organizational approaches
\end{itemize}

\entry{Research Associate \& Postdoctoral Fellow, Harvard University}{2018--2024}

\begin{itemize}
\tightlist
\item
  \textbf{Longitudinal Neuroimaging \& Reliability}: Lead analyst on
  intensive longitudinal fMRI study (50+ adolescents, 200+ sessions)
  examining stress and psychopathology mechanisms; developed and
  evaluated multilevel models of neural activation reliability
\item
  \textbf{Intensive Longitudinal Data \& Measurement Development}:
  Designed digital phenotyping protocol (stress, sleep, digital
  communication, physical activity over 3+ years); validated new
  self-report and behavioral task measures for adolescent populations,
  leading to 3 publications (NIMH R37-MH119194, \$9M)
\item
  \textbf{Multi-Site Longitudinal Analyses}: Analyzed fMRI and extensive
  survey data from 500+ adolescents across 4 sites on cognitive control
  and reward (NIMH U01-MH109589, \$17.1M); conducted systematic,
  pre-registered measurement invariance testing
\item
  \textbf{Methodological Mentoring}: Provided research design
  consultation and statistical guidance to 8 graduate students, 15
  postbaccalaureate RAs, and 9 postdoctoral fellows across 2 lab groups
\end{itemize}

\entry{Graduate Research Fellow, University of Oregon}{2012--2018}

\begin{itemize}
\tightlist
\item
  Developed hierarchical Bayesian reinforcement learning model examining
  adolescent social motives as causes of health-risking behavior
\item
  Collected and analyzed data from 300+ participants, including
  foster-care-involved adolescents requiring enhanced IRB protections,
  parental consent/youth assent procedures, and mandated reporting
  protocols
\item
  Conducted multilevel modeling and SEM of longitudinal personality and
  fMRI task data
\item
  Validated new behavioral task and self-report measures
\end{itemize}

\entry{Research Coordinator, Stanford University}{2009--2012}

\begin{itemize}
\tightlist
\item
  Coordinated Simons Foundation-funded study examining sleep problems
  and autism spectrum disorder symptoms
\item
  Site coordinator for registered clinical trial evaluating novel PET
  biomarker of cerebral amyloid in dementia patients
\end{itemize}

\section{Publications \& Scientific
Dissemination}\label{publications-scientific-dissemination}

\emph{See
\href{https://scholar.google.com/citations?user=ZzQlngkAAAAJ&hl=en}{Google
Scholar} for full list of 25+ peer-reviewed articles}

\textbf{First/Lead Author}

Flournoy, J. C., Lee, C. S., Wu, M., \& Hicks, C. M. (2025). No Silver
Bullets: Why Understanding Software Cycle Time is Messy, Not Magic.
\emph{Empirical Software Engineering}, 30, 103.
doi:10.1007/s10664-025-10735-w

Flournoy, J. C., Bryce, N. V., Dennison, M. J., et al.~(2024). A
precision neuroscience approach to estimating reliability of neural
responses during emotion processing: Implications for task-fMRI.
\emph{NeuroImage}, 285, 120503.

Flournoy, J. C., Vijayakumar, N., Cheng, T. W., et al.~(2020). Improving
practices and inferences in developmental cognitive neuroscience.
\emph{Developmental Cognitive Neuroscience}, 100807.

\textbf{Collaborative Publications}

Bryce, N., Flournoy, J.C., Moreira, J.F.G., et al.~(2021). Brain
parcellation selection: An overlooked decision point with meaningful
effects on individual differences in resting-state functional
connectivity. \emph{NeuroImage}, 118487.

Ludwig, R. M., Flournoy, J. C., \& Berkman, E. T. (2019). Inequality in
personality and temporal discounting across socioeconomic status?
Assessing the evidence. \emph{Journal of Research in Personality}, 81,
79--87.

Matta, T. H., Flournoy, J. C., \& Byrne, M. L. (2018). Making an unknown
unknown a known unknown: Missing data in longitudinal neuroimaging
studies. \emph{Developmental Cognitive Neuroscience}, 33, 83--98.

\section{Teaching, Mentoring \& Capacity
Building}\label{teaching-mentoring-capacity-building}

\textbf{Workshops \& Training}

\begin{itemize}
\tightlist
\item
  ABDC Workshop Instructor (2021): \emph{Modeling Developmental
  Change}---Data science tools, SEM theory and hands-on tutorial
\item
  UC Adolescence Consortium Presenter (2021): \emph{Why and How to Care
  About Covariates in Longitudinal Data}
\item
  Lifespan Informatics \& Neuroimaging Center, UPenn Seminar (2021):
  \emph{Scientific Practice in Developmental Cognitive Neuroscience}
\item
  Institute for Technology in Psychiatry, McLean Hospital (2019):
  \emph{Machine Learning as a Tool for Diagnosis and Theory Testing}
\item
  Graduate Statistics Teaching Assistant, University of Oregon
  Department of Psychology (2014--2015)
\item
  Co-lead, UO R Club: Led tutorials and consultations on R programming,
  data manipulation, simulation, and multilevel modeling (2014--2016)
\item
  Contributor, UO Bayes Club: Bayesian inference workshops using R and
  JAGS (2014--2016)
\end{itemize}

\textbf{Mentorship \& Multi-Project Management}

\begin{itemize}
\tightlist
\item
  Provided methodological consultation, study design guidance, and
  statistical training to 32+ researchers (graduate students, postbac
  RAs, postdocs) across 2 major research groups
\item
  Managed concurrent analytic workflows across multiple funded projects
  (NIMH R37 and U01 grants at Harvard; multiple consulting clients
  simultaneously)
\item
  Mentored development of research proposals, analysis plans, and
  manuscript preparation
\item
  Trained research staff in human subjects ethics, IRB protocols, data
  quality assurance, and open science practices
\end{itemize}

\textbf{Dissemination for Technical \& Non-Technical Audiences}

\begin{itemize}
\tightlist
\item
  Guest, The Stack Overflow Podcast (2025): Communicated statistical
  methodology concepts for a broad developer audience
\item
  Synthesized research findings into accessible reports and
  presentations for stakeholders across varying levels of quantitative
  training
\item
  Production Assistant, American Masters (PBS, 2005)
\end{itemize}

\section{Open Science \& Methodological
Infrastructure}\label{open-science-methodological-infrastructure}

\textbf{Reusable Tools \& Templates for Research Teams (R Packages, CRAN
\& GitHub)}

\begin{itemize}
\tightlist
\item
  \textbf{riclpmr}: Generate syntax for random intercept cross-lagged
  panel models---enables collaborators to implement complex longitudinal
  designs without writing model code from scratch
\item
  \textbf{curvish} (alpha): Bayesian analysis and visualization of GAM
  smooths using 1st and 2nd derivatives---supports flexible
  nonparametric estimation with Bayesian uncertainty quantification
\item
  \textbf{scorequaltrics}: Retrieve and score survey data from Qualtrics
  using CSV templates---standardizes survey data processing workflows
  across research teams
\item
  \textbf{verse-cmdstan}: Docker container providing reproducible
  Bayesian computing environments (R + Stan)---eliminates setup barriers
  for collaborators and ensures identical environments across laptops,
  CI, and HPC clusters
\end{itemize}

\textbf{Open Data, Reproducibility \& Quality Assurance}

\begin{itemize}
\tightlist
\item
  Apply test-driven development (TDD) to all research workflows: writing
  tests for data processing pipelines, model specifications, and
  analytic code before implementation---ensuring analytic quality
  assurance and verification
\item
  Contributor to Open Science Framework; publicly shared data and
  analysis code for multiple studies
\item
  Active practitioner of pre-registration, open science practices, and
  transparent statistical reporting
\end{itemize}

\section{Education \& Professional
Training}\label{education-professional-training}

\entry{Ph.D., Psychology, University of Oregon}{2018}

Focus: Computational methods, Bayesian hierarchical modeling,
developmental neuroscience

\entry{M.S., Psychology, University of Oregon}{2014}

\entry{B.A., Cognitive Science, University of California, Berkeley}{2005}

\textbf{Specialized Training}

\begin{itemize}
\tightlist
\item
  Advanced Bayesian Models for the Social Sciences \textbar{} ICPSR
  Summer Program in Quantitative Methods, 2015
\item
  Causal Inference for the Social Sciences \textbar{} ICPSR Summer
  Program in Quantitative Methods, 2015
\item
  Python Programming and Neuroinformatics \textbar{} Neurohackweek, 2016
\end{itemize}

\section{Academic Appointments}\label{academic-appointments}

Associate of the Department of Psychology (Courtesy Appointment)
\textbar{} Harvard University, 2024--Present

\section{Awards \& Recognition}\label{awards-recognition}

\begin{itemize}
\tightlist
\item
  The Sackler Scholar Programme in Psychobiology Research Grant, 2019
\item
  Gary E. Smith Summer Professional Development Award, 2015
\item
  Clarence and Lucille Dunbar Scholarship, 2014
\end{itemize}




\end{document}
